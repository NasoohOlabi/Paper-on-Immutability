%PLEASE COPY THE FILE AND EDIT THE COPY, DO NOT EDIT THIS FILE!!!

\documentclass[12pt,twoside]{article}
\usepackage[a4paper,width=150mm,top=30mm,bottom=30mm,bindingoffset=10mm]{geometry}
\linespread{1}
\usepackage[utf8]{inputenc} %Standard diacritics in Romance languages (accents, umlauts)
\usepackage{times} %Uses Times New Roman font
\usepackage{wasysym}
\usepackage{diagbox}
\usepackage{graphicx}
\graphicspath{ {./images/} }
\usepackage[style=authoryear-icomp,natbib=true,sortcites=true]{biblatex}% natbib=true so we can use natbib commands with biblatex
% Also: authoryear-icomp is used so that when you cite the same author with different years, you get "according to Herberger (2002, 2004)" rather than "according to Herberger (2002), Herberger (2004)"
\addbibresource{refs.bib}

\usepackage{xpatch}

\usepackage[tiny]{titlesec}
\titleformat{\subsection}{}{\thesubsection}{1em}{\itshape}
\titleformat{\subsubsection}{}{\thesubsubsection}{1em}{\itshape}
\titlelabel{\thetitle.\quad}

\usepackage{fancyhdr}
\pagestyle{fancy}
\fancyhead{}
\fancyhead[LE]{\thepage \hspace{3mm} \small Isogloss YEAR, ISSUE/NR}
\fancyhead[RE]{\small Author}
\fancyhead[LO]{\small Title of paper}
\fancyhead[RO]{\small Isogloss YEAR, 1/2 \normalsize \thepage}
\fancyfoot{}
%\setcounter{page}{23} %This sets the initial page at a number other than 1 (in this case 23).

\fancypagestyle{first}{
	\fancyhead{}
	\fancyhead[L]{\small ISSN 2835-4138 (digital)\\https://doi.org/10.5565/rev/isogloss.\\{}}
	\fancyhead[R]{\small Isogloss YEAR, ISSUE/NR\\PAGES\\{}}
	\fancyfoot{}}
	%don't worry about the headers, we will compile them

\newcommand{\pref}[1]{(\ref{#1})} % If you use \ref{xx}, the reference in the text appears without parentheses: "as we see in 1, ..." instead of "as we see in (1)...". So we create a new command: instead of \ref, call \pref (Parentheses REFerence) which specifies that any cross references to xx appear in parentheses.

\usepackage{tipa} %for IPA
\usepackage{phonrule} %for phonological rules
\usepackage[nocenter]{qtree} %for trees
\usepackage{gb4e} %for examples and glossing

\usepackage[normalem]{ulem} %STRIKETHROUGH TEXT

\usepackage{authblk,etoolbox}
\renewcommand\Authfont{\Large}
\renewcommand\Affilfont{\normalsize}

\makeatletter
% patch \maketitle so that it doesn't center
\patchcmd{\@maketitle}{center}{flushleft}{}{}
\patchcmd{\@maketitle}{center}{flushleft}{}{}
% patch \maketitle so that the font size for the title is normal
\patchcmd{\@maketitle}{\LARGE}{\normalsize}{}{}

\def\maketitle{{%
		\renewenvironment{tabular}[2][]
		{\begin{flushleft}}
			{\end{flushleft}}
		\AB@maketitle}}
\makeatother

\title{\Huge{Title}}
\author{Anonymous} \affil{anon \\email} 
\setlength{\affilsep}{1pt}
\date{}

%IF THERE ARE TWO AUTHORS:
%\title{\Huge{Patterns of syntactic microvariation: the case of European Portuguese}}
%\author{Noam Chomsky} \affil{Massachusetts Institute of Technology\\noam.chomsky@mit.edu} 
%\author{Howard Lasnik} \affil{University of Maryland\\howard.lasnik@umd.edu}
%\setlength{\affilsep}{1pt}
%\date{}

\begin{document}


\maketitle

\thispagestyle{first}

\vspace{0.5cm}

\hfill Received: 20-01-2034 

\hfill Accepted: 21-01-2034

\hfill Published: 22-01-2034

\vspace{1cm}

\noindent \textbf{How to cite} Leave blank

\vspace{1.5cm}

\noindent \textbf{Abstract}
\begin{center}
 	\line(1,0){430}
\end{center}
\vspace{-0,3cm}
\noindent Here's is the abstract of the paper.

\vspace{5mm}

\noindent \textbf{Keywords:} keyword1; kw2; kw3; kw4; kw5

\vspace{4mm}
\begin{center}
	\line(1,0){430}
\end{center}

%\begin{center}
%	\textbf{Table of Contents}
%\end{center}

%\begin{large}
%\begin{center}
	%\begin{tabular}{c c}
%		1. Section 1 & 4. Section 4\\
%		2. Section 2 & 5. Section 5\\
%		3. Section 3 & 6. Section 6
%	\end{tabular}
%\end{center}
%\end{large}

\section{Introduction}


Lorem ipsum dolor sit amet, consectetur adipiscing elit. According to  \citet{Chomsky:01}, Sed consectetur eu justo eget suscipit. Integer vitae lacinia velit. Quisque fermentum, mi pulvinar blandit elementum, enim turpis gravida tortor, vel maximus est dui nec mauris. Donec et mattis augue. Phasellus quis lacus velit. In aliquet metus at lacus fermentum, eu cursus purus sodales. Ut eleifend leo a dui maximus placerat. Suspendisse vitae tincidunt nulla, ac varius ligula. Donec libero sem, pharetra sagittis rutrum posuere, imperdiet et libero. \textcite[122]{Chomsky:95:The}



Here's an example:
\begin{exe}
\ex
\begin{xlist}


\item \label{8} \gll Paul a 	repeint 		les chaises\\ 
Paul has 	painted-\textsc{sg.m}	the chairs-\textsc{pl.f}\\
‘Paul has painted the chairs’ \hspace{2cm}   

\item  *Paul a repeintes les chaises \label{8b}

\end{xlist}

\ex \label{9}
\gll Paul les 		a 	repeintes \\ 
	 Paul them-\textsc{pl.f}	has 	painted-\textsc{pl.f} \\
	‘Paul repainted them’			


\end{exe}

According to Chomsky: ‘We have considered four functional categories: T, C, D, and Agr. The first three have Interpretable features providing "instructions" at either or both interface levels. Agr does not; it consists of -Interpretable formal features only. We therefore have fairly direct evidence from interface relations about T, C, and D, but not Agr. Unlike the other functional categories, Agr is present only for theory-internal reasons’						\citep[321]{Chomsky:95:The}

Vestibulum ante ipsum primis in faucibus orci luctus et ultrices posuere cubilia Curae; Fusce dui dolor, dictum nec dui at, maximus mollis tellus. Praesent auctor ut sapien in luctus. Etiam malesuada, felis sit amet faucibus ultrices, orci metus ultricies massa, vitae sagittis velit magna vel risus. Aliquam molestie ipsum elit. Aenean ut tellus erat. Cras vitae elit nibh. Lorem ipsum dolor sit amet, consectetur adipiscing elit. Sed egestas orci eget consequat fringilla. Cras tortor ex, maximus at fermentum et, imperdiet et velit.

Here's how you draw a simple tree:
\begin{exe}
\ex \label{tree}

\Tree [.Agr_{s}P [. ] [.Agr_{s}' [.Agr_{s} ][.I/TP [. ] [.I/T' [.I ] [.(NegP) [. ] [.Neg' [.Neg ] [.Agr_{o}P  [. ] [.Agr_{o}' [.Agr_{o} ] [.VP ] ] ] ] ] ] ] ]]

%\Tree [.S [.NP ] [.VP ] ]

\end{exe}
Etiam sit amet arcu euismod, iaculis neque sit amet, commodo mauris. Curabitur vel congue neque, nec aliquam elit. Proin efficitur ante in mi tristique, et consectetur mauris dignissim. Proin vitae placerat dolor. Vivamus sollicitudin felis accumsan mauris vestibulum dapibus. Maecenas ultricies sem ut magna finibus, ut dapibus lorem maximus. Morbi ut mauris tellus. Nunc tincidunt in sapien non porttitor. Mauris molestie, velit et aliquet ultrices, turpis mi placerat enim, et aliquam tellus leo sit amet odio.


Here's how you draw a table: \\


\noindent \begin{tabular}[t]{| c| c| c| c|}
\hline
A & B & C & D \\
\hline
\textsc{f}[ ] &  x &  \textsc{F}[3] &  z \\
\hline
\textsc{num}[ ] & y  &  \textsc{}{num}[PL] & l  \\
\hline
  &  &  \textsc{case} [ ] &  \\
 \hline
\end{tabular}
\\

or without borders: \\

\begin{tabular}[t]{l l l l l}
 bla & bla & bla & bla & bla\\
 a  & b  & c  & d   & e \\
 a &  b & c & d & e \\
 
\end{tabular} \\



Phasellus eget erat at leo finibus varius. Suspendisse in nunc eget neque feugiat tristique. Sed sagittis eget quam vitae dignissim. Nunc vitae sodales eros. Nullam eu quam id nisi egestas hendrerit. Ut a pulvinar felis. Morbi ut ullamcorper urna. Sed vitae sapien et nisl malesuada hendrerit. Praesent scelerisque volutpat lacus, quis imperdiet velit tempus ac. Phasellus tincidunt non est non vulputate. Fusce at est nunc.\\

A simple numbered list:

\begin{xlist}
    \item Matching is feature identity.
    \item D(P) is the sister of P.
    \item Locality reduces to "closest c-command" "
\end{xlist}


an itemized list

\begin{itemize}
    \item bla 
    \item bla
    \item bla
    
\end{itemize}

\subsection{Subsectionheader}
\subsubsection{Subsubsection header}
blabla


% The following syntax, in tandem with xpatch package, removes the brackets in the references list. It also removes the dot after the line.
% See https://tex.stackexchange.com/a/428193/169121
\xpatchbibmacro{date+extradate}{%
	\printtext[parens]%
}{%
\setunit{\addperiod\space}%
\printtext%
}{}{}


\section*{Acknowledgements}
The author wishes to thank Blabla and bla

\printbibliography


\end{document}